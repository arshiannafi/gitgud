\documentclass{beamer}

\usepackage{tikz}
\usetikzlibrary{shapes.geometric, arrows, scopes}

\title{\Large \bf Git Gud}
\author{Annafi A., Balmer T., Hassam H., Platonov D., Senekal R., Spalding-Jamieson J., and Sauren}
\institute{Students of UBC, UVic, and SFU} % not endorsed
\date{June 25, 2017}

\begin{document}


\begin{frame}
	%\maketitle
	\titlepage
\end{frame}

\begin{frame}

\frametitle{Git Gud: Overview}
	\begin{description}
		\item{-} Analyze git commit subjects and bodies in order to return a rating.
		\item{-} Use categorization fitting to determine if your commit sucks.
		\item{-} Extensive collection and categorization of training data.
		\item{-} Training data includes commits from: Microsoft, Google, Facebook, Linux
		\item{-} Training data was obtained through GitHub, although other sources could easily be added,
			such as Visual Studio Team Foundation Services and Bitbucket.
	\end{description}

\end{frame}

\begin{frame}

\frametitle{High-Level Program Flow}
	\tikzstyle{subprogram} = [rectangle, minimum width=2cm, minimum height=.5cm,
		text centered, draw=black, fill=gray]
	\tikzstyle{subprocess} = [rectangle, rounded corners, minimum width=2cm, minimum height=.5cm,
		text centered, draw=black, fill=white]
	\tikzstyle{external} = [rectangle, rounded corners, dashed, minimum width=2cm, minimum height=.5cm,
		text centered, draw=black, fill=yellow]
	\tikzstyle{poss} = [rectangle, rounded corners, dashed, minimum width=2cm, minimum height=.5cm,
		text centered, draw=white, fill=orange]
	\tikzstyle{connection} = [thick,-]
	\begin{tikzpicture}[node distance = .7cm]
		\node (server) [subprogram] {
			Server
			\begin{tikzpicture}
				\node (parser) [subprocess] {Data Parser/Computer};
				\node (server-interface) [subprocess, below of=parser] {Server Interface};
				\node (abs) [subprocess, left of=server-interface, xshift=-2.5cm] {Abstraction Layer};
				\node (gathering) [subprocess, below of=abs, xshift=-2cm] {Data Gathering/Storing};
				\node (network) [external, below of=server-interface] {Network};
				\node (locations) [poss, below of=gathering, xshift=1.5cm] {Github, Team Foundation Services, Bitbucket,etc.};
				\draw [connection] (parser) -- (server-interface);
				\draw [connection] (server-interface) -- (abs);
				\draw [connection] (abs) -| (gathering);
				\draw [connection] (gathering) -- (locations);
				\draw [connection] (network) -- (server-interface);
			\end{tikzpicture}
		};
		
		\node (client) [subprogram, below of=server, yshift=-2.4cm] {
			Client
			\begin{tikzpicture}
				\node (network) [external] {Network};
				\node (fetch) [subprocess, below of=network] {Client Interfacing};
				\node (hook) [subprocess, below of=fetch] {Hook(ed) Tool};
				\node (tools) [poss, below of=hook] {Git from terminal, Git plugins, etc.};
				\draw [connection] (hook) -- (fetch);
				\draw [connection] (network) -- (fetch);
				\draw [connection] (tools) -- (hook);
			\end{tikzpicture}
		};
			
		%\node (server) [fit = (parser)(server-interface)(abs)(gathering), header = Server] {};
		%\node (client) [fit = (fetch)(hook), header = Client] {};
		
	\end{tikzpicture}

\end{frame}

\begin{frame}

\frametitle{Training: Data Labelling}
	\begin{description}
		\item{-} Uses definitive rules in combination with Fork, Watch, Stars, and keywords in order to calculate quality.
		\item{-} Definitive rules should give the model a baseline for what is important (because they are important).
		\item{-} Definitive rules include:
		\begin{description}
			\item{-} ($\Theta(1)$) Capitalized subject
			\item{-} ($\Theta(1)$) Max 50 characters in subject
			\item{-} ($\Theta(1)$) No period at end of subject
			%technically nlog m, where m is size of dictionary, but m is fixed
			\item{-} ($\Theta(n)$) Verbs in present tense 
			\item{-} ($\Theta(n)$) Uses bullet points in body
			%technically nlog m, where m is size of dictionary, but m is fixed
			\item{-} ($\Theta(n)$) No swear words
		\end{description}
		\item{-} Keywords are found using the Mircosoft Cognitive Services to check topic consistency for commits.
	\end{description}

\end{frame}

\begin{frame}

\frametitle{Training Data Flow}
	\tikzstyle{info} = [rectangle, minimum width=.5cm, minimum height=.5cm, text centered, draw=black, fill=violet]
	\tikzstyle{process} = [rectangle, rounded corners, minimum width=2cm, minimum height=.5cm, text centered, draw=black, fill=white]
	\tikzstyle{flow} = [thick,->]
	\tikzstyle{joined} = [-]
	\begin{tikzpicture}[node distance = 1.7cm]
		\node (metadata) [info] {Forks, Stars, Watches};
		\node (commit) [info, right of=metadata, xshift=2cm] {Commit Message};
		\node (rater) [process, below of=commit] {Sample Data Rater};
		\node (exp) [info, below of=rater] {Expected Values};
		\node (model) [process, right of=rater, xshift=1.5cm, minimum width = 2.25cm, minimum height = 1.75cm] {Model};
		\draw [joined] (commit) -- (metadata);
		\draw [flow] (commit) -- (rater);
		\draw [flow] (metadata) |- (rater);
		\draw [flow] (rater) -- (exp);
		\draw [flow] (exp) -| (model);
		\draw [flow] (commit) -| (model);
	\end{tikzpicture}

\end{frame}

\end{document}
